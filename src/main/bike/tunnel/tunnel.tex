\chapter{Fahrradtunnel}
Gerade im dicht verbauten Grazer Stadtgebiet rund um das Zentrum ist oft nicht ausreichend Platz für große Fahrradinfrastruktur vorhanden. Oft ist es möglich, auf Kosten anderer Verkehrsteilnehmer die Radinfrastruktur auszubauen, doch das stößt an seine Grenzen.

Eine der Lösungen dafür sind unterirdische Tunnel ausschließlich für den Fahrrad-Verkehr.

Die Vorteile liegen auf der Hand:
\begin{itemize}
    \item Schnelle Direktverbindung
    \item Fahrräder werden nicht durch andere Verkehrsteilnehmer ausgebremst
    \item Andere Verkehrsteilnehmer werden nicht durch Fahrräder ausgebremst
    \item Tunnel sind eine wetterfeste Option
\end{itemize}

Die Nachteile sind jedoch auf der anderen Seite ebenfalls offensichtlich:
\begin{itemize}
    \item Sehr teuer
    \item Unflexibel -- Zu- und Ausfahrt sind nur an bestimmten Stellen möglich
    \item Jede Zufahrt braucht eine flache, raumintensive Rampe
\end{itemize}

Tunnel sind deswegen nur an ausgewählten Stellen möglich und sinnvoll, wo einerseits auf der Oberfläche wirklich nicht ausreichend Platz ist, und auf der anderen Seite ein hohes Potential gegeben ist.

Überlegenswert ist auch, ob man die Tunnel -- je nach Ort, noch breiter ausbaut, um sie auch für Fußgänger als attraktive Alternative zu öffnen.

\section{Route 1: Fahrradtunnel Universitäten}
Folgende Route schlagen wir als Hauptverbindung im Grazer Osten in Tunnelform vor:

\begin{itemize}
    \item Schulzentrum St. Peter
    \item Neue Technik
    \item Alte Technik
    \item Lichtenfelsgasse
    \item KFU Hauptgebäude (Kreuzung mit Route 2)
    \item Kreuzschwestern-Park
    \item Wirtschaftskammer/Campus02
\end{itemize}

\begin{figure}
    \includegraphics[width=0.8\textwidth]{main/bike/tunnel/uni/linie1}
    \centering
    \caption[Verlauf Linie 1]{Verlauf der Linie 1 zwischen den Zufahrten}
\end{figure}

Zwischen den einzelnen Universitäten findet oft besonders reger Radverkehr statt, und auf dieser Achse ist aufgrund der Bauweise der Stadt oft besonders wenig Platz für Radinfrastruktur.

Dieser Tunnel stellt eine Achse Nord-Süd dar und ist mit einer Kreuzung sowohl mit dem Stadtzentrum als auch dem Landeskrankenhaus verbunden.

\subsection{Portal Schulzentrum St. Peter}
Die Zufahrt \index{Schulzentrum St. Peter} soll (vorläufig) der südlichste Punkt der Linie 1 sein und damit sowohl selbst ein großes Einzugsgebiet als auch eine gute Anbindung an den weiteren Verkehr haben. Beides ist gegeben; nicht umsonst ist die Haltestelle Schulzentrum St. Peter schon aktuell ein wichtiger Verkehrsknotenpunkt für den öffentlichen Verkehr.

Die Zufahrt soll möglichst nah an der Straßenbahnhaltestelle Schulzentrum St. Peter liegen.

\subsubsection{Variante 1 -- Haltestelle St. Peter Schulzentrum}

\begin{figure}
    \includegraphics[width=0.8\textwidth]{main/bike/tunnel/uni/zufahrt_inffeld}
    \centering
    \caption[Zufhart St. Peter Schulzentrum]{Vorgeschlagener Ort der Zufahrt in den Tunnel der Linie 1}
\end{figure}


Die Zufahrt selbst wird einen Teil des Grünbereichs neben der Haltestelle auffressen.

Durch diese Zufahrt wird das gesamte Schulzentrum erschlossen, zu dem unter anderem die \index{Berufsschule St. Peter}, der \index{Campus Inffeldgasse} der Technischen Universität Graz, das \index{BRG Petersgasse} und das \index{WIKU BRG Graz} gehören.

Damit die Zufahrt besser an den Verkehr angebunden ist, sind hier weitere Maßnahmen nötig:

\paragraph{Fahrradgarage St. Peter Schulzentrum}
Direkt im Bereich der Zufahrt sollte eine Parkgarage für Fahrräder unterirdisch errichtet werden. Diese Garage kann sowohl dazu dienen, das Fahrrad dort abzustellen, während man eine der Ausbildungsstätten besucht, als auch als Park-and-Ride-Lösung für Personen, die von hier aus mit dem Bus oder der Straßenbahn weiterfahren wollen.

\paragraph{Anschluss Berufsschule}
Es soll eine Möglichkeit geschaffen werden, von der Zufahrt direkt auf das Gelände der Berufsschule St. Peter zu gelangen. Dafür ist die Errichtung eines weiteren (kurzen) Radweges nötig, der auch eine sichere Querung der \index{Hans-Brandstetter-Gasse} erlaubt.

\paragraph{Straßenquerung Petersgasse}
Ein signifikanter Teil der Benutzer der Zufahrt werden aus der \index{Inffeldgasse} kommen bzw. in die Inffeldgasse weiterfahren wollen, um einerseits den Campus Inffeldgasse der Technischen Universität Graz zu erreichen, als auch den Anschluss an die Radroute \index{Neufeldweg}, die am westlichen Ende der Inffeldgasse ihren Anfang nimmt. Dafür ist eine sichere Querung der \index{Petersgasse} nötig. Da die Petersgasse eine wichtige Route für den motorisierten Individualverkehr ist, gibt es hier inhärentes Konfliktpotential.

\paragraph{Alternative: Zwei Ausfahrten}
Um das Konfliktpotential in der Petersgasse zu umgehen, wäre es auch denkbar, eine Unterführung unter der Straße anzubieten und eine zweite Ausfahrt am Anfang der Inffeldgasse zu bauen, sodass die Zufahrt St. Peter Schulzentrum letztlich aus zwei unterschiedlichen Zufahrten besteht. Das hat natürlich den Nachteil höherer Kosten, löst aber den Konflikt in der Petersgasse komplett auf.

\subsubsection{Variante 2 -- Inffeldgasse}
\begin{figure}
    \centering
    \includegraphics[width=0.8\textwidth]{main/bike/tunnel/uni/zufahrt_inffeld_v2}
    \caption[Zufahrt Inffeldgasse]{In dieser Variante liegt die Zufahrt parallel zur Inffeldgasse.}
\end{figure}

In dieser Variante liegt die Rampe gänzlich in der Inffeldgasse. Diese Variante hat einige Vorteile, aber auch Nachteile. Positiv sticht hervor, dass hier deutlich mehr Platz vorhanden ist, die Rampe kann sehr lang und damit flach ausfallen. Der Platz, der hier verlorengeht, müsste von den Parkplätzen, die hier zu beiden Seiten der Inffeldgasse liegen, genommen werden. Dafür müssten etwa 40 Parkplätze aufgelöst werden, die allesamt der Technischen Universität zuzurechnen sind.

\paragraph{Straßenquerung Petersgasse}
Bei dieser Variante ist der Anschluss an den Campus Inffeldgasse bedeutend besser -- das Portal liegt direkt auf der Privatstraße des Campus. Will man jedoch zur Berufsschule oder zu der auf der anderen Seite der Petersgasse liegenden Haltestelle, braucht man wieder eine Straßenquerung, die auch in dieser Variante eventuell angepasst werden müsste.

\paragraph{Direktverbindung BRG Petersgasse und WIKU BRG Graz}
Die Ausfahrt liegt in der Luftlinie nur wenige Meter von diesen beiden Schulen entfernt, es gibt jedoch keine vernünftige Direktverbindung, sodass ein Radfahrer einen großen Umweg über das Gelände der Technischen Universität bis in die Sandgasse nehmen muss, um dorthin zu kommen. Der bevorzugte Verlauf dieser Verbindung ist in Abbildung \ref{fig:direktverbindung_brg_petersgasse} dargestellt.

\begin{figure}
    \centering
    \includegraphics[width=0.6\textwidth]{main/bike/tunnel/uni/direktverbindung_brg_petersgasse}
    \caption[Direktverbindung Portal -- BRG Petersgasse]{An dieser Stelle müsste eine Direktverbindung vom Portal zu den beiden Schulen BRG Petersgasse und WIKU BRG Graz sichergestellt werden.}
    \label{fig:direktverbindung_brg_petersgasse}
\end{figure}

\paragraph{Fahrradgarage Inffeldgasse}
Auch bei dieser Variante sollte eine Fahrradgarage mitgedacht werden. Der Bedarf für Fahrradstellplätze ist gerade bei der Technischen Universität sehr groß, auf diese Art könnte also oberirdisch Platz gespart werden. Liegt die Garage außerdem richtig, ist auch ein zweiter Fußgänger-Aufgang von der Garage aus zur Haltestelle St. Peter Schulzentrum denkbar, was die Garage für Umsteiger zwischen Fahrrad und Öffentlichem Verkehr sehr attraktiv machen kann.

\subsubsection{Verlauf zwischen St. Peter Schulzentrum und Neuer Technik}
Je nach genauer Platzierung der Rampen und der Ausfahrtsrichtung bei der Neuen Technik ergibt sich ein gerader Verlauf des Tunnels wie in Abbildung \ref{fig:uni_verlauf1} dargestellt.

\begin{figure}
    \centering
    \includegraphics[width=0.8\textwidth]{main/bike/tunnel/uni/verlauf1}
    \caption[Tunnelverlauf zwischen Inffeldgasse und Neuer Technik]{Tunnelverlauf zwischen Inffeldgasse und Neuer Technik}
    \label{fig:uni_verlauf1}
\end{figure}

\subsection{Zufahrt Neue Technik}
Die nächste Zufahrt soll im Bereich der \index{Neuen Technik} liegen. Einerseits ist der Radpendelverkehr zwischen Inffeldgasse und Neuer Technik traditionell immer sehr hoch, andererseits gibt es dafür oberirdisch keine vernünftige Möglichkeit, weswegen es zu zahlreichen Konflikten zwischen Autofahrern und Radfahrern auf der \index{Petersgasse} kommt.

\subsubsection{Variante 1 -- Ecke Schörgelgasse}

\begin{figure}
    \centering
    \includegraphics[width=0.8\textwidth]{main/bike/tunnel/uni/zufahrt_neue_technik}
    \caption[Zufahrt Neue Technik]{Ort der Zufahrt Neue Technik auf dem Vorplatz der Biochemie}
\end{figure}

\paragraph{Anschluss}
Von hier aus ist ein guter Anschluss an das weitere Radwegenetz gegeben. Über die \index{Schörgelgasse} gelangt man direkt zum \index{Dietrichsteinplatz} und über den anschließenden Radweg in die \index{Koperinikusgasse}. Auf der anderen Straßenseite befindet sich mit dem \index{Sacré Coeur} eine weitere größere Schule.

\paragraph{Fahrradgarage}
An dieser Zufahrt bietet sich eien unterirdische Fahrradgarage sehr an, da hier die Stellplätze an der Oberfläche begrenzt sind. Sowohl das Geländer der Neuen Technik als auch des Sacré Coeur bieten Stellmöglichkeiten an, die sind jeweils aber sehr begrenzt.

\paragraph{Umgestaltung der Kreuzung Schörgelgasse/Petersgasse/Mandellstraße}
\begin{figure}
    \centering
    \includegraphics[width=0.6\textwidth]{main/bike/tunnel/uni/kreuzung1}
    \caption[Petersgasse/Mandellstraße mit Kreuzung Schörgelgasse]{Die Petersgasse mit den Einmündungen der Schörgelgasse ist aktuell bereits ein unübersichtlicher Bereich.}
\end{figure}

Die Mandellstraße geht bei der ersten Einmündung der Schörgelgasse in die Petersgasse über. Diese Kreuzung ist bisher schon recht unübersichtlich, da hier zahlreiche unterschiedliche Verkehrsteilnehmer in unterschiedliche Richtungen abbiegen. Es gibt einen Zebra-Streifen, mehrere Fahrradspuren und die zwei sehr nah beieinanderliegenden Kreuzungen Felix-Dahn-Platz/Mandellstraße/Petersgasse/Schörgelgasse und Petersgasse/Schörgelgasse.

\paragraph{Option: Unterirdische Querung der Petersgasse}
Als Option wäre eine unterirdische Querung der Petersgasse denkbar, evtl. ein direkter Zugang zum Sacré Coeur von der Fahrradgarage aus, sodass hier die Querung der Petersgasse nicht mehr nötig ist, wenn man von der Zufahrt zum Sacré Coeur möchte.

\paragraph{Nachteile}
Ein gravierender Nachteil dieser Stelle ist, dass der gesamte Vorplatz des Gebäudes, der bisher eine schöne Anlage ist, dafür aufgebraucht würde, und auch die Aussicht auf das Gebäude würde gestört werden. An dieser Stelle ist sehr wenig Platz, weswegen eine Rampe vergleichsweise steil wäre.

\subsubsection{Variante 2 -- Parkplatz auf dem Universitätsgelände}
Eine zweite Möglichkeit für die Platzierung der Rampe wäre hier der Parkplatz zwischen den großen Universitätsgebäuden wie in Abbildung \ref{fig:zufahrt_neue_technik_v2} abgebildet. Bei dieser Variante würden etwa 35 Parkplätze verloren gehen, die allesamt der Technischen Universität Graz zuzurechnen wären.

\begin{figure}
    \centering
    \includegraphics[width=0.8\textwidth]{main/bike/tunnel/uni/zufahrt_neue_technik_v2}
    \caption[Zufahrt Parkplatz Neue Technik]{Bei dieser Variante würde die Rampe über dem alten Parkplatz platziert werden.}
    \label{fig:zufahrt_neue_technik_v2}
\end{figure}

\paragraph{Ausfahrtsrichtung}
Die Ausfahrt ist hier prinzipiell in beide Richtungen möglich. Ist die Ausfahrt Richtung Osten, dann steht der Radfahrer direkt an der Petersgasse, wo es kein gutes Weiterkommen mehr gibt, weil dort keine Radinfrastruktur mehr vorhanden ist. Geschieht die Ausfahrt Richtung Westen, dann steht der Radfahrer direkt im Gelände der Technischen Universität Graz und muss dieses teilweise durchqueren, um zu weiteren Radwegen (z.B. über den Anschluss Schörgelgasse) zu gelangen.

\paragraph{Fahrradgarage}
Eine unterirdische Garage für die Fahrräder ist in dieser Variante genauso wichtig wie in der Variante 1, je nach Ausfahrtsrichtung könnte in unterirdischer Zugang zum Sacré Coeur aber nicht mehr möglich sein.

\paragraph{Nachteile}
Bei dieser Variante ist zwar deutlich mehr Platz für die Rampe, und es gehen "nur" Parkplätze verloren. Jedoch müssen -- je nach Ausfahrtsrichtung -- weitere Schritte unternommen werden, um die Ausfahrt gut an das restliche Netz anzubinden.

\subsection{Zufahrt Alte Technik}
Ein weiterer häufig benutzter Weg, der an der Oberfläche über kaum Fahrradinfrastruktur verfügt, ist die Verbindung zwischen der Neuen Technik und der \index{Alten Technik}.

\paragraph{Ort der Zufahrt}
\begin{figure}
    \centering
    \includegraphics[width=0.6\textwidth]{main/bike/tunnel/uni/zufahrt_alte_technik}
    \caption[Zufahrten Alte Technik]{Bei der Alten Technik wird ein Teil der Technikerstraße zur Rampe umfunktioniert.}
    \label{fig:zufahrt_alte_technik}
\end{figure}

Hier bietet sich die Rampe im oberen Verlauf der Technikerstraße direkt neben dem Gebäude der Alten Technik an, wie in Abbildung \ref{fig:zufahrt_alte_technik} dargestellt. Der MIV könnte hier dann nicht mehr direkt von der \index{Technikergasse} in die \index{Gartengasse} fahren, sondern müsste einen kleinen Umweg über die \index{Morellenfeldgasse} machen. Außerdem würden hier etwa 5 Parkplätze in der Blauen Zone verloren gehen.

\paragraph{Verbesserung Radwegenetz}
Unabhängig davon, dass hier eine Zufahrt zum Tunnel besteht, braucht der Bereich rund um die Neue Technik eine Verbesserung des Radwegenetzes. Die Rechbauerstraße sowie alle seine Seitengassen sind ein typischer Konfliktbereich für Fahrräder und Autos. Hier sollte überlegt werden, Parkstreifen für Autos aufzulösen und daraus Radwege zu machen, insbesondere in der Rechbauerstraße, der \index{Lessingstraße}, der \index{Alberstraße}, der Gartengasse, der Morellenfeldgasse, der \index{Wastiangasse} und der \index{Haydngasse}.

\paragraph{Fahrradgarage}
Eine unterirdische Fahrradgarage bietet sich in diesem Bereich besonders an, da das Gebiet besonders dicht verbaut ist und auf der Oberfläche sehr wenig Platz dafür ist. Dadurch kann an der Oberfläche Platz, der aktuell als Fahrradabstellplatz genutzt wird, wieder anderweitig verwendet werden.

\paragraph{Alternative: Verkehrsberuhigter Bereich}
Sollte der Alternativvorschlag umgesetzt werden, das Viertel um die Alte Technik zu einem Verkehrsberuhigten Bereich umzugestalten (wie in Abbildung \ref{fig:alte_technik_vbb} angedeutet), dann würde diese Zufahrt zum Fahrradtunnel sogar noch mehr an Bedeutung gewinnen und könnte etwas zentraler verortet sein (zum Beispiel im Dreieck Morellenfeldgasse, Technikerstraße und Rechbauerstraße). %TODO: Cross-Referenz

\begin{figure}
    \centering
    \includegraphics[width=0.8\textwidth]{main/bike/tunnel/uni/verkehrsberuhigter_bereich}
    \caption[Verkehrsberuhigkter Bereich Alte Technik]{Vorschag eines Verkehrsberuhigten Bereichs um die Alte Technik, der die Zufahrt Alte Technik noch wichtiger machen würde.}
    \label{fig:alte_technik_vbb}
\end{figure}

\subsection{Zufahrt Lichtenfelsgasse}
Die nächste Zufahrt befindet sich im Uni-Park ist ein wichtiger Anschluss zum \index{BG/BRG Lichtenfelsgasse} und den gesamten westlichen Bereich zwischen Leonhardstraße und Elisabethstraße.

\paragraph{Ort der Zufahrt}
\begin{figure}
    \centering
    \includegraphics[width=0.8\textwidth]{main/bike/tunnel/uni/zufahrt_lichtenfelsgasse}
    \caption[Zufahrt Lichtenfelsgasse]{Zufahrt zur Lichtenfelsgasse im Uni-Park hinter der Universität für Musik und Darstellende Kunst}
\end{figure}

Die Zufahrt soll sich im Uni-Park hinter der Universität für Musik und Darstellende Kunst verbinden, da hier der einzige Platz in dieser Gegend dafür ist.

\paragraph{Anschluss}
Wichtiges Einzugsgebiet dafür sind das BG/BRG Lichtenfels, die Universität für Musik und Bildende Kunst sowie das Theater im Palais. Um einen sicheren Zugang zum Lichtenfels-Gymnasium zu ermöglichen, sollte hier überprüft werden, ob ein Übergang über die Lichtenfelsgasse angepasst werden muss. Außerdem sollte durch den Uni-Park ein Radweg mit Anschluss zur Lichtenfelsgasse und im Süden in die \index{Brandhofgasse} geplant werden.

\paragraph{Fahrradgarage}
An dieser Position sollte eine Fahrradgarage mittlerer Größe angedacht werden. Das spart einerseits Platz auf der Oberfläche, bietet aber auch genügend Abstellplätze für all jene, die zur Uni oder der Schule fahren wollen.

\paragraph{Option: Umgestaltung der Lichtenfelsgasse}
Die Hauptzufahrt zur 

\subsection{Mögliche Verlängerung nach Süden}
Von der Zufahrt Schulzentrum St. Peter aus ist eine Verlängerung nach Süden denkbar mit folgenden weiteren Zufahrten:
\begin{itemize}
    \item ORF-Park
    \item Wohnpark St. Peter
    \item Messendorf
    \item Raaba
\end{itemize}

\subsection{Mögliche Verlängerung nach Norden}
Von der Zufahrt Wirschaftskammer aus ist eine weitere Verlängerung in Richtung Norden denkbar mit folgenden weiteren Zufahrten:
\begin{itemize}
    \item Ortweinschule
\end{itemize}

Von dort aus ist das Radwegenetz auch oberirdisch gut möglich und teilweise bereits ausgebaut.
\section{Route 2: Fahrradtunnel Innenstadt -- LKH}
Eine weitere mögliche Route mit potentiell hohem Fahrradverkehrsaufkommen und wenig Platz auf der Oberfläche ist eine Verbindung vom LKH in die Innenstadt, von wo eine Direktverbindung zum Murradweg möglich sein kann und soll.

Folgende Zufahrten sind hier geplant:
\begin{itemize}
    \item LKH
    \item Universität Hauptgebäude (Kreuzung mit Route 1)
    \item Stadtpark Murpromenade
\end{itemize}


