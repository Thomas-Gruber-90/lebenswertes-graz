\chapter{Allgemeine Überlegungen}
In diesem Kapitel sammeln wir generelle Überlegungen und Konzepte zur Radinfrastruktur in Graz. Dabei geht es darum, wo welche Infrastruktur bevorzugt werden sollte, wie das passieren kann und warum.

\section{Ausweichroute Plabutschtunnel}
Der \index{Plabutschtunnel} ist eine essentielle Verkehrsader für den motorisierten Individualverkehr sowie für den Güterverkehr. Er ist Teil der \emph{A9 Phyrn-Autobahn} und damit eine wichtige Umfahrung des Grazer Stadtgebietes in Nord-Süd-Richtung.

Erfahrungsgemäß ist der Tunnel jedoch immer wieder gesperrt. Das führt oft zu einem Verkehrschaos im Grazer Stadtgebiet, da der gesamte Verkehr nun durch Graz muss.

Wir erkennen die \glqq{}Ausweichroute\grqq{}\footnote{Wiener Straße, Bahnhofgürtel, Kärntner Straße, Weblinger Kreisverkehr} für den Plabutschtunnel durch Graz als solche an und sind uns bewusst, dass eine Behinderung des Autoverkehrs auf dieser Route unbedingt vermieden werden muss. Im Falle einer Sperrung des Tunnels muss sichergestellt werden, dass der Ausweichverkehr nach wie vor möglichst flüssig durch Graz fließen kann, um ein Verkehrschaos zu verhindern.

Insbesondere sollen auf dieser Route keine Fahrspuren verschwinden und der Autoverkehr soll hier die höchste Priorität haben.